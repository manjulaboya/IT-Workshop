\documentclass{article}
 \usepackage{fancyhdr}
 \usepackage{amsthm}
\usepackage[a4paper, margin=2in]{geometry}
\usepackage{graphicx}
\usepackage{amsmath}
\usepackage{amssymb}
\usepackage[colorlinks=true,urlcolor=blue]{hyperref}
\hypersetup{citecolor=blue,
pdfborderstyle={/S/U/W 2},
}
\fancyhead[R]{\thepage}
\makeatletter
\renewcommand\section{\@startsection{section}{1}{\z@}%
	{-3.5ex \@plus -1ex \@minus -.2ex}%
	{2.3ex \@plus.2ex}%
	{\normalfont\large}}
	\makeatother
	\begin{document}
	% \hspace{2cm}
	\newtheorem{theorem}{Theorem}
	\newtheorem{conjecture}{Conjecture}
 \textbf{THE 18.821 MATHEMATICS PROJECT LAB REPORT\newline
 	      [REPLACE THIS WITH YOUR OWN SHORT\newline
           DESCRIPTIVE TITLE!]}
 \centering\section*{X. BURPS, P. GURPS }	          
 	         
\text{Abstract. This is a \LaTeX{} template for 18.821, which you can }\newline use for your own reports.	
 \centering\section{INTRODUCTION}
 This brief document shows some examples of the use of L A TEX and
 indicates some special features of the Math Lab report style. The\textcolor{blue}{\underline{ \href{http://stellar.mit.edu/S/course/18/sp13/18.821/}{course website}}}
   contains links to several \LaTeX{} manuals.
 
      End the introduction by describing the contents of the paper sec­
 tion by section, and which team member(s) wrote each of them. For
 instance, Section 6 discusses referencing, and is written by P. Gurps.
 \section{\LaTeX{} EXAMPLES}
 Here are some ways of producing mathematical symbols. Some are
 package whic h
 pre-defined either in L A TEX or in the AMS a
 h n this document
 n
 loads. For instance, sums and integrals,\[\sum_{i=1}^{n} 1=n,\int_0^n x\;\mathrm{d}x=\tfrac{n^2}{2}\]
 We’ve defined a few other symbols at the start of the document, for
 instance $\mathbb{N,Q,Z,R}$ You can make marginal notes for you1rself or your
 co-authors like this:
 Unfinished here?
 If you want to typeset equations, there are many choices, with or
 without numbering:\[\int_0^1 x\;\mathrm{d}x=\tfrac{1}{2},\] 
 \raggedright or\[\sum_{i=1}^{\infty} i=,\tfrac{-1}{2}\] or\[1-1+1-...=\frac{1}{2}\]
 \date{February 10, 2013} 
 \centering\section*{X. BURPS, P. GURPS}
 \begin{figure}[h]
 	\centering
 	\includegraphics[width=0.5\textwidth]{itpic.png}
 	\caption{My first .pdf figure. }
 \end{figure}
 If you want a number for an equation, do it like this:
\begin{align} \lim_{n \to \infty}\sum_{k=1}^{n}\frac{1}{k^2}=\frac{\pi}{6} \end{align}
This can then be referred to as (1), which is much easier than keeping
track of numbers by hand. To group several equations, aligning on the
= sign, do it like this:
\[x_1 + 2x_2 + 3x_3 = 7\]  
\[y = mx + c\]  
\[= 4x - 9.\]
You can easily embed hyperlinks into the output .pdf document:
\textcolor{blue}{\underline{\href{http://stellar.mit.edu/S/course/18/sp13/18.821/}{click here}}} for example.
\section{IMAGES}
Figure 1 is an example of a .pdf image put into a floating environ­
ment, which means LaTeX will draw it wherever there’s enough space
left in your manuscript. Look at the .tex original to see how to insert
a figure like this.
\section{THEOREMS AND SUCH}
   An example of a “conjecture environment” is given below, in Con­
jecture 4.1. Theorems, lemmas, propositions, definitions, and such all
use the same command with the appropriate name changed. In fact,
 \newpage \section*{THE 18.821 REPORT}if you look at the top of this .tex file, you can see where we’ve defined these environments.\\
 \begin{conjecture}[Vaught's Conjecture]Let $T$ be a countable complete theory. If $T$ has fewer than $2^{\aleph_0}$ many countable models (up to isomorphism), then it has countably many countable models.
 \end{conjecture}
 \begin{theorem}
 	When it rains it pours.
 \end{theorem}
 \begin{proof}
 	Well, yes.
 \end{proof}
\section{FILETYPES USED BY LATEX}
You will write your text as a .tex file using any text editor (though WYSIWYG editors are troublesome). Traditionally one then runs \LaTeX{} and obtains a .dvi file, which can be viewed on the screen using a dvi viewer. To include images, and then prepare the file for printing or submission, one typically translates the .dvi into either .ps (Postscript) or .pdf (AdoPDF). 
\par{}\hspace{0.5 cm}Your report will be submitted as a .pdf document. The \texttt{pdflatex}
command produces a .pdf file directly from a .tex file. This command
works well with included .pdf files, but does not handle .eps files.
An .eps file can be converted to a .pdf file by viewing it and saving
as a .pdf file, or by \texttt{2pdf filename.eps,} which produces
\texttt{filename.pdf.} Under MikTeX with WinEdt, all necessary commands
will appear under “Accessories” in the WinEdt menu.
Finally, Matlab can be made to produce .eps files by typing\\
\texttt{print -deps filename}\\
at the prompt.     
\section{QUOTING SOURCES}
 In your work, keep notes of the literature you’ve used, including websites. Cite the references you use; failure to do so constitutes plagiarism. Every bibliography item should be referenced somewhere in the paper. Quote as precisely as possible:[~\cite{itref},pages 76--78] rather than~\cite{itref}.~\cite{itref2} was a useful background reference, too.
  \begingroup \renewcommand{\refname}{REFERENCES}
  \begin{thebibliography}{9}
  	\bibitem{itref}
  	Gurps, P., \emph{Care and feeding of maths professors}. Cambridge Univ. Press, 2008.
  	\bibitem{itref2}
  	Burps, X. \emph{Terrors and errors of project lab}. \emph{Journal of Wildlife and Conservation} 21 (2008), 112--134.
  \end{thebibliography}
  \endgroup
  \appendix
\section*{ APPENDEX}
Appendices are useful for putting in code or data.\newline
 
 \begin{flushleft}\textbf{MITOpenCourseWare}\\\textcolor{blue}{\underline{\href{http://ocw.mit.edu}{http://ocw.mit.edu}}}\end{flushleft}
 \vspace{1cm}
 \begin{flushleft}\textbf{18.821 Project Laboratory in Mathematics}\\Spring 2013 \end{flushleft}
  \vspace{1cm} 
 For information about citing these materials or our Terms of Use, visit:\textcolor{blue}{\underline{\href{http://ocw.mit.edu/terms}{ http://ocw.mit.edu/terms.}}}
\end{document}
